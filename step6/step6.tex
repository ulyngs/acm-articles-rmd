\documentclass[sigchi, review]{acmart}

\usepackage{booktabs} % For formal tables

% Copyright
%\setcopyright{none}
%\setcopyright{acmcopyright}
%\setcopyright{acmlicensed}
%\setcopyright{rightsretained}
%\setcopyright{usgov}
%\setcopyright{usgovmixed}
%\setcopyright{cagov}
\setcopyright{licensedcagov}
%\setcopyright{cagovmixed}
%\setcopyright{licensedothergov}

% DOI
\acmDOI{10.475/123_4}

% ISBN
\acmISBN{123-4567-24-567/08/06}

%Conference
\acmConference[WOODSTOCK'97]{ACM Woodstock conference}{July 1997}{El
  Paso, Texas USA}
\acmYear{1997}
\copyrightyear{2016}

\acmPrice{15.00}


\usepackage{amsthm}
\newtheorem{theorem}{Theorem}[section]
\newtheorem{lemma}{Lemma}[section]
\theoremstyle{definition}
\newtheorem{definition}{Definition}[section]
\newtheorem{corollary}{Corollary}[section]
\newtheorem{proposition}{Proposition}[section]
\theoremstyle{definition}
\newtheorem{example}{Example}[section]
\theoremstyle{definition}
\newtheorem{exercise}{Exercise}[section]
\theoremstyle{remark}
\newtheorem*{remark}{Remark}
\newtheorem*{solution}{Solution}
\begin{document}
\title{This is the Greatest and Best Paper in the World (Tribute)}
\titlenote{Produces the permission block, and
  copyright information}
\subtitle{Extended Abstract}
\subtitlenote{The full version of the author's guide is available as
  \texttt{acmart.pdf} document}

\author{Ben Trovato}
\authornote{Dr.~Trovato insisted his name be first.}
\orcid{1234-5678-9012}
\affiliation{%
  \institution{Institute for Clarity in Documentation}
  \streetaddress{P.O. Box 1212}
  \city{Dublin}
  \state{Ohio}
  \postcode{43017-6221}
}
\email{trovato@corporation.com}

\author{G.K.M. Tobin}
\authornote{The secretary disavows any knowledge of this author's actions.}
\affiliation{%
  \institution{Institute for Clarity in Documentation}
  \streetaddress{P.O. Box 1212}
  \city{Dublin}
  \state{Ohio}
  \postcode{43017-6221}
}
\email{webmaster@marysville-ohio.com}

\author{Lars Th{\o}rv{\"a}ld}
\authornote{This author is the
  one who did all the really hard work.}
\affiliation{%
  \institution{The Th{\o}rv{\"a}ld Group}
  \streetaddress{1 Th{\o}rv{\"a}ld Circle}
  \city{Hekla}
  \country{Iceland}}
\email{larst@affiliation.org}

\author{Valerie B\'eranger}
\affiliation{%
  \institution{Inria Paris-Rocquencourt}
  \city{Rocquencourt}
  \country{France}
}
\author{Aparna Patel}
\affiliation{%
 \institution{Rajiv Gandhi University}
 \streetaddress{Rono-Hills}
 \city{Doimukh}
 \state{Arunachal Pradesh}
 \country{India}}
\author{Huifen Chan}
\affiliation{%
  \institution{Tsinghua University}
  \streetaddress{30 Shuangqing Rd}
  \city{Haidian Qu}
  \state{Beijing Shi}
  \country{China}}

\author{Charles Palmer}
\affiliation{%
  \institution{Palmer Research Laboratories}
  \streetaddress{8600 Datapoint Drive}
  \city{San Antonio}
  \state{Texas}
  \postcode{78229}}
\email{cpalmer@prl.com}

\author{John Smith}
\affiliation{\institution{The Th{\o}rv{\"a}ld Group}}
\email{jsmith@affiliation.org}

\author{Julius P.~Kumquat}
\affiliation{\institution{The Kumquat Consortium}}
\email{jpkumquat@consortium.net}

% The default list of authors is too long for headers.
\renewcommand{\shortauthors}{B. Trovato et al.}


\begin{abstract}
This is the greatest and best abstract in the world. Tribute.
\end{abstract}

%
% The code below should be generated by the tool at
% http://dl.acm.org/ccs.cfm
% Please copy and paste the code instead of the example below.
%
\begin{CCSXML}
<ccs2012>
 <concept>
  <concept_id>10010520.10010553.10010562</concept_id>
  <concept_desc>Computer systems organization~Embedded systems</concept_desc>
  <concept_significance>500</concept_significance>
 </concept>
 <concept>
  <concept_id>10010520.10010575.10010755</concept_id>
  <concept_desc>Computer systems organization~Redundancy</concept_desc>
  <concept_significance>300</concept_significance>
 </concept>
 <concept>
  <concept_id>10010520.10010553.10010554</concept_id>
  <concept_desc>Computer systems organization~Robotics</concept_desc>
  <concept_significance>100</concept_significance>
 </concept>
 <concept>
  <concept_id>10003033.10003083.10003095</concept_id>
  <concept_desc>Networks~Network reliability</concept_desc>
  <concept_significance>100</concept_significance>
 </concept>
</ccs2012>
\end{CCSXML}

\ccsdesc[500]{Computer systems organization~Embedded systems}
\ccsdesc[300]{Computer systems organization~Redundancy}
\ccsdesc{Computer systems organization~Robotics}
\ccsdesc[100]{Networks~Network reliability}


\keywords{ACM proceedings, \LaTeX, text tagging}

\begin{teaserfigure}
  \includegraphics[width=\textwidth]{sampleteaser}
  \caption{This is a teaser}
  \label{fig:teaser}
\end{teaserfigure}


\maketitle

\section{Introduction}\label{introduction}

``Tribute'' is the first single of Tenacious D's self-titled debut
album\citep{Cowan1988, Whittaker2016-time-info}. It was released July
16, 2002. The song is a tribute to what Gass and Black refer to as ``The
Greatest Song in the World'' (often confused as the song's title), which
Tenacious D themselves came up with, but have since forgotten. It was
released as a downloadable track for Rock Band in addition to appearing
as a playable track for Guitar Hero Live.\citep{Hume1748}

\section{History}\label{history}

Tribute was the first song Black and Gass played live as Tenacious D.
The song, like many other songs that were recorded on Tenacious D, was
originally performed on their short-lived HBO TV series. During earlier
performances of this song Kyle Gass played the opening to ``Stairway to
Heaven''. The two songs are both in A minor and have very similar chord
progressions, and critics have said the songs sound
alike.\citep{Cowan1988, Gluck2005, Wardak2002} \textbf{The maturation of
the song over time is shown in Figure \ref{fig:tribute-plot}.}

\begin{figure}
\includegraphics[width=0.98\columnwidth]{step6_files/figure-latex/tribute-plot-1} \caption{This is how great Tribute gets over time}\label{fig:tribute-plot}
\end{figure}

\begin{figure*}
\includegraphics[width=0.98\textwidth]{step6_files/figure-latex/two-col-tribute-plot-1} \caption{This is a two-column plot of how great Tribute gets over time}\label{fig:two-col-tribute-plot}
\end{figure*}

\subsection{Synopsis}\label{synopsis}

The song chronicles the band members' encounter with a demon who demands
the duo play ``the best song in the world'' or have their souls eaten.
Having nothing to lose from trying, they play ``the first thing that
came to our heads'', and it ``just so happened to be the best song in
the world.''\citep{Wardak2002}

\begin{table}

\caption{\label{tab:table-iris}The favorite iris' of Tenacious D.}
\centering
\begin{tabular}[t]{rrrr}
\toprule
Sepal.Length & Sepal.Width & Petal.Length & Petal.Width\\
\midrule
5.1 & 3.5 & 1.4 & 0.2\\
4.9 & 3.0 & 1.4 & 0.2\\
4.7 & 3.2 & 1.3 & 0.2\\
4.6 & 3.1 & 1.5 & 0.2\\
5.0 & 3.6 & 1.4 & 0.2\\
\addlinespace
5.4 & 3.9 & 1.7 & 0.4\\
4.6 & 3.4 & 1.4 & 0.3\\
5.0 & 3.4 & 1.5 & 0.2\\
4.4 & 2.9 & 1.4 & 0.2\\
4.9 & 3.1 & 1.5 & 0.1\\
\bottomrule
\end{tabular}
\end{table}

Given the ``Stairway to Heaven'' interlude in the original TV series
version, along with the similarity of the chord progression in both
songs, Tribute at first implies that the best song in the world is
indeed that song. However, the lyrics make clear that Tribute sounds
nothing like the song they came up with to please the demon; as Black
describes: ``And the peculiar thing is this my friends: The song we sang
on that fateful night, it didn't actually sound anything like this
song.''

\bibliographystyle{ACM-Reference-Format}
\bibliography{my-bibliography}

\end{document}
